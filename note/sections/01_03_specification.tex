\subsection{Требования к проектируемому программному средству}
\label{sec:analysis:specification}

По результатам изучения предметной области \cite{ddd_quickly} и обзора существующих систем-аналогов сформулируем требования к проектируемому программному средству.

\subsubsection{} Назначение проекта
\label{sec:analysis:specification:purpose}

Назначением проекта является разработка программного средства, упрощающего основные задачи слежения за финансами пользователя. Приложение должно быть интуитивно понятным, а скорость взаимодействия должна быть максимальной.

\subsubsection{} Основные функции
\label{sec:analysis:specification:functions}

Программное средство должно поддерживать следующие основные фун\-к\-ции:

\begin{itemize}
	\item работа с категориями (создание, изменение, удаление, выбор типа, валюты, названия);
	\item добавление операции в категории;
	\item уведомление пользователя о приближении к заданному порогу трат;
	\item вывод статистики по категориям;
	\item очистка статистики за последний месяц в каждом новом месяце.
\end{itemize}

\subsubsection{} Требования к входным данным
\label{sec:analysis:specification:inputs}

Входные данные для программного средства должны быть представлены в виде вводимого пользователем с помощью клавиатуры текста и выбора доступных опций интерфейса чата приложения Telegram.

Должны быть реализованы проверки вводимых данных на корректность с отображением информации об ошибках в случае их некорректности.

\subsubsection{} Требования к выходным данным
\label{sec:analysis:specification:outputs}

Выходные данные программного средства должны быть представлены посредством отображения информации с помощью различных элементов интерфейса чата приложения Telegram.

\subsubsection{} Требования к временным характеристикам
\label{sec:analysis:specification:timing}

Производительность программно-аппаратного комплекса должна обеспечивать следующие временные характеристики: время реакции не запрос пользователя не должно превышать 1 секунды при минимальной скорости соединения 1 МБит/с. Допускается невыполнение данного требования в случае, когда невозможность обеспечить заявленную производительность обусловлена объективными внешними причинами.

\subsubsection{} Требования к аппаратному обеспечению серверной части
\label{sec:analysis:specification:server_requirments}

ЭВМ, на которой должна функционировать серверная часть программного средства, должна обладать следующими минимальными характеристиками:

\begin{itemize}
	\item процессор Intel Core i5 с тактовой частотой 2 ГГц;
	\item жесткий диск объемом 100 Гб;
	\item оперативная память 4 Гб;
	\item сетевая карта Ethernet 100 МБит/с.
\end{itemize}

Telegram Bot API требует наличия шифрованного HTTPS соединения, которое может быть обеспечено бесплатными утилитами, либо установкой дополнительного ПО наподобие Nginx.

Сформулированные требования позволят осуществить успешное проектирование и разработку программного средства.
