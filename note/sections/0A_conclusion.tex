\sectioncentered*{Заключение}
\addcontentsline{toc}{section}{Заключение}

Основными целями данного дипломного проекта были проектирование и разработка программного средства <<Чат-бот для контроля за персональными расходами>>. Предметной областью дипломного проекта является упрощение выполнения некоторых задач процесса слежения за персональными расходам. Был проведен поиск существующих программных средств этого рода, по его результатам был сделан вывод о не существовании полных аналогов. Предложенное программное средство должно объединить в себе достоинства всех конкурентов, а также быть лишенным их недостатков.

На основании проведенного анализа предметной области были выдвинуты требования к программному средству. В качестве технологий были выдвинуты наиболее современные существующие на данный момент средства, широко применяемые в индустрии. Спроектированное программное средство было успешно протестировано на соответствие спецификации функциональных требований. 

Таким образом, исходя из анализа предметной области и того факта, что на рынке не существует полных аналогов предложенного программного средства, можно сделать вывод о целесообразности проектирования и разработки программной системы. Результаты, полученные в ходе выполнения технико-экономического обоснования подтвердили данный вывод.

Разработано программное средство, которое поддерживает следующие функции:
\begin{itemize}
	\item работа с категориями (создание, изменение, удаление, выбор типа, валюты, названия);
	\item добавление операции в категории;
	\item уведомление пользователя о приближении к заданному порогу трат;
	\item вывод статистики по категориям;
	\item очистка статистики за последний месяц в каждом новом месяце.
\end{itemize}

В качестве будущего улучшения программного продукта предлагаются несколько вариантов, описанных ниже:

\begin{itemize}
	\item Внедрение искусственной нейронной сети \cite{networks} для улучшения понимания более человеческих запросов пользователя, а не только одинаковых команд. Например, вместо команды /expense, пользователь может написать <<50 долларов на наушники>>, и бот должен интерпретировать это как операция с 50 долларами в категории техника, либо наушники. В настоящее время, нейронный сети сделали большой шаг вперед, поэтому их внедрение в подобного рода программные средства может быть очень полезным.
	\item Расширение функциональности путем внедрения поддержки варианта использования, когда пользователь, после совершения покупок в магазине, фотографирует чек и отсылает фотографию боту. Бот сам поймет какие покупки были совершены и на какие суммы, и добавит операции в соответствующие категории.
	\item Более наглядная статистика в виде графиков, таблиц, диаграмм. На основании имеющихся данных будут выведены подсказки, чего стоит покупать меньше в этом месяце, а также статистика по месяцам, товары каких категорий покупались чаще, какие – реже.
\end{itemize}
