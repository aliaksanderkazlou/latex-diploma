\section{Тестирование и проверка работоспособности программного средства}
\label{sec:testing}

Тестирование программного обеспечения -- процесс исследования, испытания программного продукта, имеющий своей целью проверку соответствия между реальным поведением программы и её ожидаемым поведением на конечном наборе тестов, выбранных определенным образом~\cite{kulikov_testing}.

Тестирование можно классифицировать по очень большому количеству признаков. Основные виды классификации включают следующие~\cite{kulikov_testing}:

\begin{enumerate}
	\item по запуску кода на исполнение:
	\begin{enumerate}
		\item статическое тестирование -- без запуска программного средства;
		\item динамическое тестирование -- с запуском;
	\end{enumerate}
	\item по степени автоматизации:
	\begin{enumerate}
		\item ручное тестирование -- тестовые случаи выполняет человек;
		\item автоматизированное тестирование -- тестовые случаи частично или полностью выполняет специальное инструментальное средство;
	\end{enumerate}
	\item по принципам работы с приложением:
	\begin{enumerate}
		\item позитивное тестирование -- все действия с приложением выполняются строго в соответствии с требованиями без недопустимых действий или некорректных данных;
		\item негативное тестирование -- проверяется способность приложения продолжать работу в критических ситуациях недопустимых действий или данных.
	\end{enumerate}
\end{enumerate}

Прежде  всего,  были  разработаны  модульные  тесты,  предназначенные для  тестирования  отдельных  модулей  без  привязки  их  к  другим  модулям. Использование модульных тестов позволяло точно определить место возникновения ошибки и достаточно быстро исправлять ее. Модульное тестирование проводилось с использованием фреймворка NUnit. 

После того как были протестированы отдельные модули, необходимо было  протестировать  корректность  взаимодействия  модулей  между  собой. 

На протяжении всего времени разработки программного средства осуществлялось регрессионное тестирование. Сутью регрессионного тестирования является воспроизведение ранее выполнявшихся тестов, чтобы удостовериться  в  том,  что  добавление  новой  функциональностиили  исправление ошибокне привело к появлению новых ошибок.

Кроме  автоматизированного  тестирования  также  проводилось  ручное тестирование. Был  разработан  ряд  тест-кейсов, проверяющихкорректность работы приложения. В таблице~\ref{table:testing:testcases}  приведен набор тестовых случаев, использовавшихся при тестировании программного средства.

Тестирование проводилось на следующем оборудовании:

\begin{itemize}
	\item персональный компьютер в качестве API-сервера;
	\item облачный хостинг от DigitalOcean в качестве API-сервера;
	\item персональный сматрфон в качестве клиента;
	\item персональный компьютер в качестве клиента;
\end{itemize}

Конфигурация персонального компьютера:

\begin{itemize}
	\item ОС Microsoft Windows 10 Pro x64 1709 Build 16299.431;
	\item процессор Intel Core i7-4700HQ @ 2.40GHz;
	\item ОЗУ 16Гб DDR3 1600MHz;
	\item SSD Samsung SSD 840 EVO 250GB;
	\item графический адаптер Geforce GT 750M;
\end{itemize}

Конфигурация облачного сервера:

\begin{itemize}
	\item ОС Ubuntu 16.04.4 x64;
	\item процессор 1vCPU;
	\item ОЗУ 2Гб;
	\item SSD 50GB;
\end{itemize}

Конфигурация персонального смартфона:

\begin{itemize}
	\item ОС Android 7.1.2;
	\item процессор Octa-core MAX 1.4GHz;
	\item ОЗУ 3Гб;
\end{itemize}

\begin{longtable}{|>{\raggedright}m{0.02\textwidth}|
		 >{\raggedright}p{0.3\textwidth}|
		 >{\raggedright}p{0.3\textwidth}|
		 >{\raggedright\arraybackslash}p{0.30\textwidth}|} 
	\caption{Классы и методы блока работы через протокол HTTPS}
	\label{table:testing:testcases}\\

	\hline 
	\centering № & \centering Название тест-кейса & \centering Шаги & \centering\arraybackslash Ожидаемый результат \endfirsthead

	\caption*{Продолжение таблицы \ref{table:design:server:db}}\\\hline
	\centering 1 & \centering 2 & \centering 3 & \centering\arraybackslash 4 \\\hline \endhead

	\hline
	\centering 1 & \centering 2 & \centering 3 & \centering\arraybackslash 4 \\
	\hline

	1 &
	Старт чата с ботом &
	1. Зайти в диалог с ботом; \newline
	2. Ввести команду \emph{/start} &
	Сообщение с перечислением функций бота \\ \hline

	2 &
	Добавление категории-дохода &
	1. Зайти в диалог с ботом и ввести команду \emph{/category}; \newline
	2. Выбрать \emph{Add category}; \newline
	3. Ввести имя категории; \newline
	4. Выбрать валюту категории; \newline
	5. Выбрать \emph{income}  &
	Сообщение об успешном создании категории \\ \hline

	3 &
	Добавление категории-дохода c попыткой 
	ввести неверную валюту категории &
	1. Зайти в диалог с ботом и ввести команду \emph{/category}; \newline
	2. Выбрать \emph{Add category}; \newline
	3. Ввести имя категории; \newline
	4. Ввести любую строку кроме \emph{EUR}, \emph{BYN}, или \emph{USD}; &
	Сообщение об с уведомлением о том, что пользователь должен выбрать определенную валюту \\ \hline

	4 &
	Добавление категории c попыткой 
	ввести имя существующей категории &
	1. Зайти в диалог с ботом и ввести команду \emph{/category}; \newline
	2. Выбрать \emph{Add category}; \newline
	3. Ввести имя категории, которое уже было использовано; &
	Сообщение об с уведомлением о том, что имя должно быть уникально \\ \hline

	5 &
	Добавление категории с попыткой ввести неверный тип категории &
	1. Зайти в диалог с ботом и ввести команду \emph{/category}; \newline
	2. Выбрать \emph{Add category}; \newline
	3. Ввести имя категории; \newline
	4. Выбрать валюту категории; \newline
	5. Ввести любой тип категории, кроме \emph{Income} и \emph{Exprense}  &
	Сообщение с уведомлением о том, что пользователь может выбрать только \emph{Income} либо \emph{Exprense} категорию \\ \hline

	6 &
	Добавление категории с отменой действия &
	1. Зайти в диалог с ботом и ввести команду \emph{/category}; \newline
	2. Выбрать \emph{Add category}; \newline
	3. Ввести имя категории; \newline
	4. Выбрать валюту категории; \newline
	5. Ввести команду отмены \emph{/cancel}  &
	Сообщение об отмене действия \\ \hline

	7 &
	Добавление категории-расхода &
	1. Зайти в диалог с ботом и ввести команду \emph{/category}; \newline
	2. Выбрать \emph{Add category}; \newline
	3. Ввести имя категории; \newline
	4. Выбрать валюту категории; \newline
	5. Выбрать \emph{expense} \newline 
	6. Указать желаемый порог расходов &
	Сообщение об успешном создании категории \\ \hline

	8 &
	Добавление категории-расхода с указанием нечислового значения порога расходов &
	1. Зайти в диалог с ботом и ввести команду \emph{/category}; \newline
	2. Выбрать \emph{Add category}; \newline
	3. Ввести имя категории; \newline
	4. Выбрать валюту категории; \newline
	5. Выбрать \emph{expense} \newline 
	6. Указать желаемый порог расходов в нечисловом виде &
	Сообщение с уведомлением о том, что значение не может быть нечисловым \\ \hline

	9 &
	Добавление категории-расхода с указанием отрицательного порога расходов &
	1. Зайти в диалог с ботом и ввести команду \emph{/category}; \newline
	2. Выбрать \emph{Add category}; \newline
	3. Ввести имя категории; \newline
	4. Выбрать валюту категории; \newline
	5. Выбрать \emph{expense} \newline 
	6. Указать отрицательный желаемый порог расходов &
	Сообщение с уведомлением о том, что значение не может быть отрицательным \\ \hline

	10 &
	Добавление категории-расхода с отменой действия &
	1. Зайти в диалог с ботом и ввести команду \emph{/category}; \newline
	2. Выбрать \emph{Add category}; \newline
	3. Ввести имя категории; \newline
	4. Выбрать валюту категории; \newline
	5. Выбрать \emph{expense} \newline 
	6. Ввести команду отмены \emph{/cancel} &
	Сообщение об отмене действия\\ \hline

	11 &
	Изменение категории на категорию-доход &
	1. Зайти в диалог с ботом и ввести команду \emph{/category}; \newline
	2. Выбрать \emph{Edit category}; \newline
	3. Ввести новое имя категории; \newline
	4. Выбрать новую валюту категории; \newline
	5. Выбрать новый тип категории \newline &
	Сообщение об успешном изменении категории \\ \hline

	12 &
	Изменение категории на категорию-доход c попыткой 
	ввести неверную валюту категории &
	1. Зайти в диалог с ботом и ввести команду \emph{/category}; \newline
	2. Выбрать \emph{Edit category}; \newline
	3. Ввести имя категории; \newline
	4. Ввести любую строку кроме \emph{EUR}, \emph{BYN}, или \emph{USD}; &
	Сообщение об с уведомлением о том, что пользователь должен выбрать определенную валюту \\ \hline

	13 &
	Изменение категории c попыткой 
	ввести имя существующей категории &
	1. Зайти в диалог с ботом и ввести команду \emph{/category}; \newline
	2. Выбрать \emph{Edit category}; \newline
	3. Ввести имя категории, которое уже было использовано; &
	Сообщение об с уведомлением о том, что имя должно быть уникально \\ \hline

	14 &
	Изменение категории с попыткой ввести неверный тип категории &
	1. Зайти в диалог с ботом и ввести команду \emph{/category}; \newline
	2. Выбрать \emph{Edit category}; \newline
	3. Ввести имя категории; \newline
	4. Выбрать валюту категории; \newline
	5. Ввести любой тип категории, кроме \emph{Income} и \emph{Exprense}  &
	Сообщение с уведомлением о том, что пользователь может выбрать только \emph{Income} либо \emph{Exprense} категорию \\ \hline

	15 &
	Изменение категории с отменой действия &
	1. Зайти в диалог с ботом и ввести команду \emph{/category}; \newline
	2. Выбрать \emph{Edit category}; \newline
	3. Ввести имя категории; \newline
	4. Выбрать валюту категории; \newline
	5. Ввести команду отмены \emph{/cancel}  &
	Сообщение об отмене действия. Параметры, которые пользователь успел изменить, сохраняются в системе. \\ \hline

	16 &
	Изменение категории на категорию-расход &
	1. Зайти в диалог с ботом и ввести команду \emph{/category}; \newline
	2. Выбрать \emph{Edit category}; \newline
	3. Ввести имя категории; \newline
	4. Выбрать валюту категории; \newline
	5. Выбрать \emph{expense} \newline 
	6. Указать желаемый порог расходов &
	Сообщение об успешном создании категории \\ \hline

	17 &
	Изменение категории на категорию-расход с указанием нечислового значения порога расходов &
	1. Зайти в диалог с ботом и ввести команду \emph{/category}; \newline
	2. Выбрать \emph{Edit category}; \newline
	3. Ввести имя категории; \newline
	4. Выбрать валюту категории; \newline
	5. Выбрать \emph{expense} \newline 
	6. Указать желаемый порог расходов в нечисловом виде &
	Сообщение с уведомлением о том, что значение не может быть нечисловым \\ \hline

	18 &
	Изменение категории на категорию-расход с указанием отрицательного порога расходов &
	1. Зайти в диалог с ботом и ввести команду \emph{/category}; \newline
	2. Выбрать \emph{Edit category}; \newline
	3. Ввести имя категории; \newline
	4. Выбрать валюту категории; \newline
	5. Выбрать \emph{expense} \newline 
	6. Указать отрицательный желаемый порог расходов &
	Сообщение с уведомлением о том, что значение не может быть отрицательным \\ \hline

	19 &
	Удаление категории &
	1. Зайти в диалог с ботом и ввести команду \emph{/category}; \newline
	2. Выбрать \emph{Delete category}; \newline
	3. Ввести имя категории;  &
	Сообщение об успешном удалении категории \\ \hline

	20 &
	Удаление категории с попыткой ввести имя несуществующей категории &
	1. Зайти в диалог с ботом и ввести команду \emph{/category}; \newline
	2. Выбрать \emph{Delete category}; \newline
	3. Ввести несуществующее имя категории;  &
	Сообщение с уведомлением о том, что категорий с таким именем не найдено \\ \hline

	21 &
	Попытка ввести неподдерживаемую команду во время работы с категориями &
	1. Зайти в диалог с ботом и ввести команду \emph{/category}; \newline
	2. Ввести неподдерживаемую команду  &
	Сообщение с уведомлением о том, команда не поддерживается \\ \hline

	22 &
	Добавление операции в категорию &
	1. Зайти в диалог с ботом и ввести команду \emph{/income} или \emph{/expense}; \newline
	2. Ввести имя категории \newline
	3. Ввести сумму операции \newline
	4. Ввести дату операции &
	Сообщение об успешном добавлении операции \\ \hline

	23 &
	Добавление операции в несуществующую категорию &
	1. Зайти в диалог с ботом и ввести команду \emph{/income} или \emph{/expense}; \newline
	2. Ввести имя несуществующей категории &
	Сообщение с уведомлением о том, что категорий с таким именем не найдено \\ \hline

	24 &
	Добавление операции с попыткой указания нечисловой суммы &
	1. Зайти в диалог с ботом и ввести команду \emph{/income} или \emph{/expense}; \newline
	2. Ввести имя категории \newline
	3. Ввести нечисловую сумму операции \newline &
	Сообщение с уведомлением о том, что сумма операции должна быть числом \\ \hline

	25 &
	Добавление операции с попыткой указания отрицательной суммы &
	1. Зайти в диалог с ботом и ввести команду \emph{/income} или \emph{/expense}; \newline
	2. Ввести имя категории \newline
	3. Ввести отрицательную сумму операции \newline &
	Сообщение с уведомлением о том, что сумма операции должна быть положительным числом \\ \hline

	26 &
	Добавление операции с попыткой указания даты в неверном формате &
	1. Зайти в диалог с ботом и ввести команду \emph{/income} или \emph{/expense}; \newline
	2. Ввести имя категории \newline
	3. Ввести сумму операции \newline 
	4. Ввести дату в любом формате, кроме \emph{dd.mm.yyyy} &
	Сообщение с уведомлением о том, что дата должна быть указана в определенном формате \\ \hline

\end{longtable}

Все обнаруженные в процессе тестирования ошибкибыли устранены в процессе отладки и не воспроизводилисьв следующих версиях приложения.

Таким образом, в последней версии ошибкине обнаружены. В ходе тестирования установлено, что программное средство реализует все функции, описанные в спецификации.
