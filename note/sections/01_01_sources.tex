\subsection{Сравнение нативных приложений и ботов} 
\label{sec:analysis:botsvsnative}

Чат-боты во многом привлекательны потому, что предоставляют возможность взаимодействовать с компьютерами в более естественной форме -- вместо работы с графическим интерфейсом достаточно сказать, какая задача требует выполнения. На данный момент чат-боты не развились до такого уровня, но определенный прогресс заметен уже сейчас. Например, для решения определенных задач уже сейчас пользователь может ввести определенную команду, а компьютер поймет и сделает то, что требуется.

Одно из безусловных препятствий к повсеместной популяризации ботов -- это синтаксис, который нужно помнить для взаимодействия с ними. В конце концов развитие технологий обработки языка и искусственного интеллекта позволит решить эту проблему -- компьютер будет понимать и адекватно реагировать на большинство запросов пользователей.

Несмотря на то, что в большинстве существующих решений не содержится искусственного интеллекта, можно отметить, что ряд повседневных задач возможно решать путём переписки с автоматизированным агентом. Невзирая на то, что данные решения не отличаются сообразительностью и требуют знания определенного синтаксиса, возможность решать задачи посредством короткого обмена сообщениями делает этот процесс более легким и быстрым.

Простота разговорных интерфейсов становится главной движущей распространения ботов, но дискуссия о противостоянии ботов и приложений упускает из виду некоторые детали. Контекст использования важнее чем простота. Большая доля привлекательности чат-ботов объясняется тем, что разговорные интерфейсы обладают уникальной возможностью интеграции в контекст, в котором находится пользователь. Чат-боты особенно востребованы в задачах, когда цена смены контекста слишком высока.

Возможность дать команду в чате -- это не только проще, чем сделать это через приложение. Она обеспечивает взаимодействие с минимумом отвлекающих факторов. По этой причине чат-ботам не обязательно нужно быть проще, чем аналогичные им приложения. Они должны соответствовать контексту, в котором пользователь решает определенную задачу.

Например, для того, чтобы узнать прогноз погоды, необязательно открывать приложение погоды на своем телефоне, достаточно спросить у ассистента и получить ответ, не отрываясь от текущего контекста.

Схожим образом боты в Slack позволяют делать всё что угодно в одной коллективной среде, что проще, нежели выполнение задач в разных интерфейсах и последующий сбор результатов в одно место. Интеграция вашего сервиса со Slack в виде бота снижает барьеры к принятию, потому что пользоваться сервисом можно будет в том же разговорном потоке, в котором уже ведутся дела.

Ставка на чат-ботов предполагает, что пользователи будут проводить всё больше времени в мессенджерах, и что пользоваться нужными сервисами при помощи ботов будет проще, чем открывать специальное приложение или сайт. В пользу такого развития событий уже говорит опыт Китая, в котором 650 миллионов человек проводят подавляющее большинство времени в мессенджере WeChat: они не просто общаются между собой, но и потребляют широкий ассортимент услуг.

Процесс адаптации на данный момент продвигается медленно — пользователям требуется время, чтобы привыкнуть к взаимодействию с ботами, но уже сейчас представляется возможным сделать вывод о том, что пользовательское поведение будет меняться по мере того, как элементов искусственного интеллекта в чат-платформах будет становиться больше, а грань между ботами и приложениями размоется.

Боты превзойдут нативные приложения не по причине простоты использования, а из-за очевидной пользы от возможности встраивания в контекст, в котором пользователь уже решает свою задачу. Пользователи привыкнут разговаривать с ботами в силу удобства.