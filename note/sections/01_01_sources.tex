\subsection{Аналитический обзор литературных источников}
\label{sec:analysis:literature}

Далее приводится анализ сведений, которые влияют на формулирование требований, выбор архитектуры и дальнейшее проектирование и разработку программного средства.

\subsubsection{} Кроссплатформенность приложений
\label{sec:analysis:literature:crossplatform}

Как несколько десятилетий назад, так и в настоящее время выбор платформы является серьезным ограничением для всех последующих этапов разработки. Однако уже начали появляться технологии, которые позволяют использовать однажды написанный код на многих платформах. Сегодня все больше приложений создается сразу для нескольких платформ, а приложения, созданные изначально для одной платформы, активно адаптируются под другие \cite{habr_crossplatform}. 

Преимущества кроссплатформенной разработки:

\begin{itemize}
	\item экономия бюджета – использование одной технологии и набора графики снижает количество рабочих часов и бюджет проекта;
	\item время разработки – отсутствие уникальных элементов интерфейса и одна технологическая платформа сокращает сроки разработки;
	\item поддержка и обновление продукта – добавление функционала или исправление ошибок сразу для всех платформ;
	\item сервисы и использующие их приложения могут быть развернуты на различных платформах.
	\item единая логика приложения – логика приложения будет одинаково работать для всех платформ. Написанная и отлаженная логика содержит потенциально меньшее количество ошибок и расхождений в своей работе.
\end{itemize}

Минусы кроссплатформенной разработки:

\begin{itemize}
	\item медленная работа приложения;
	\item не используются уникальные особенности платформы;
	\item непривычный для пользователя интерфейс.
\end{itemize}

\subsubsection{} Сравнение нативных приложений и ботов
\label{sec:analysis:literature:botsvsnative}

Нельзя отрицать: чат-боты во многом привлекательны потому, что в них видят возможность сделать работу с компьютерами более естественной — вместо работы с графическим интерфейсом можно будет просто сказать, что вам нужно. До такого мы пока не дошли и дойдём лишь через некоторое время, но сегодняшние чат-боты показывают нам слабые проблески того будущего, в котором можно будет что-то сказать или написать, компьютер поймёт это и сделает, что требуется.

Способности чат-ботов через несколько лет, если сравнить их с нынешними, могут разочаровать — это подтвердит любой, кто имел дело с Alexa и Siri. Одно из безусловных препятствий к всеобщему проникновению ботов — это синтаксис, который нужно помнить для взаимодействия с ними. В конце концов развитие технологий обработки языка и искусственного интеллекта позволит решить эту проблему — компьютеру можно будет сказать почти что угодно, и он поймёт, о чём речь.

До этого пока далеко, но и нынешние боты могут быть полезными, хотя никакого искусственного интеллекта у них нет. Мы уже видим, что целый ряд повседневных задач можно обрабатывать путём переписки с автоматизированным агентом. Даже если этот бот не очень «умный» и требует знания некоторого синтаксиса, возможность делать дела посредством короткого обмена репликами с чат-ботом видится по-своему волшебной.

Простота разговорных интерфейсов может быть главным драйвером их распространения, но дискуссия о противостоянии ботов и приложений упускает кое-что из виду. Это контекст использования, и он, возможно, важнее чем простота. Большая доля привлекательности чат-ботов объясняется тем, что разговорные интерфейсы обладают уникальной возможностью интеграции в контекст, в котором мы уже находимся. Вот почему чат-боты особенно востребованы, когда цена смены контекста слишком высока.

Возможность дать команду в чате — это не только проще, чем сделать это через приложение. Она обеспечивает взаимодействие с минимумом отвлекающих факторов. По этой причине чат-ботам не обязательно нужно быть проще, чем аналогичные им приложения. Они должны соответствовать контексту, в котором вы что-то делаете.

Объясню, что имею в виду. У меня на кухне стоит Amazon Echo. Наверное, в первую очередь каждым утром я спрашиваю, какая погода. Могу я достать телефон и посмотреть прогноз на нём? Конечно, но пока я готовлю завтрак, гораздо проще сказать: «Алекса, какая сегодня погода?» Я не должен прерываться и «переключать режим».

Схожим образом боты в Slack позволяют делать всё что угодно в одной коллективной среде, что проще, нежели выполнение задач в разных интерфейсах и последующий сбор результатов в одно место. Интеграция вашего сервиса со Slack в виде бота снижает барьеры к принятию, потому что пользоваться сервисом можно будет в том же разговорном потоке, в котором уже ведутся дела.

Ставка на чат-ботов предполагает, что мы будем проводить всё больше времени в мессенджерах, и что пользоваться нужными сервисами при помощи ботов будет проще, чем открывать специальное приложение или сайт. В пользу такого развития событий уже говорит опыт Китая, в котором 650 млн человек проводят кучу времени в мессенджере WeChat: они не просто болтают с друзьями, но и потребляют широкий ассортимент услуг.

Это причина, по которой Facebook возлагает большие надежды на бот-платформу для Messenger. Продукт, который они скоро представят, скорее всего, позволит почти всем желающим поставлять контент или услуги через мессенджер. Компания идёт на этот шаг, рассчитывая на то, что в конечном итоге пользователи станут «жить» в Messenger — как в случае WeChat. Это такой манёвр, чтобы обойти iOS и Android как платформы для приложений, и сделать все сервисы доступными в Messenger.

Сначала дело может пойти медленно — пользователям потребуется время, чтобы привыкнуть к взаимодействию с ботами, но не трудно представить, что пользовательское поведение будет меняться по мере того, как элементов искусственного интеллекта в чат-платформах будет становиться больше, а грань между ботами и приложениями размоется.

Если боты победят, это произойдёт не по причине простоты использования, а из-за очевидной пользы от возможности встраивания в контекст, в котором мы уже работаем или общаемся. Мы привыкнем разговаривать с ботами в силу удобства.

\subsubsection{} Обзор подходов взаимодействия в сети
\label{sec:analysis:literature:restsoap}

Самыми популярными на данный момент подходами являются REST и SOAP.

Различия в подходах описаны ниже:

\begin{itemize}
	\item SOAP активно использует XML для кодирования запросов и ответов, а также строгую типизацию данных, гарантирующую их целостность при передаче между клиентом и сервером. С другой стороны, запросы и ответы в REST могут передаваться в ASCII, XML, JSON или любых других форматах, распознаваемых одновременно и клиентом, и сервером. Кроме того, в модели REST отсутствуют встроенные требования к типизации данных. В результате пакеты запросов и ответов в REST имеют намного меньшие размеры, чем соответствующие им пакеты SOAP.
	\item В модели SOAP уровень передачи данных протокола HTTP является «пассивным наблюдателем», и его роль ограничивается передачей запросов SOAP от клиента серверу с использованием метода POST. Детали сервисного запроса, такие как имя удаленной процедуры и входные аргументы, кодируются в теле запроса. Архитектура REST, напротив, рассматривает уровень передачи данных HTTP как активного участника взаимодействия, используя существующие методы НТТР, такие как GET, POST, PUT и DELETE, для обозначения типа запрашиваемого сервиса. Следовательно, с точки зрения разработчика, запросы REST в общем случае более просты для формулирования и понимания, так как они используют существующие и хорошо понятные интерфейсы HTTP.
	\item Модель SOAP поддерживает определенную степень интроспекции, позволяя разработчикам сервиса описывать его API в файле формата Web Service Description Language (WSDL, язык описания веб-сервисов). Создавать эти файлы довольно сложно, однако это стоит затраченных усилий, поскольку клиенты SOAP могут автоматически получать из этих файлов подробную информацию об именах и сигнатурах методов, типах входных и выходных данных и возвращаемых значениях. С другой стороны, модель REST избегает сложностей WSDL в угоду более интуитивному интерфейсу, основанному на стандартных методах HTTP, описанных выше.
	\item В основе REST лежит концепция ресурсов, в то время как SOAP использует интерфейсы, основанные на объектах и методах. Интерфейс SOAP может содержать практически неограниченное количество методов; интерфейс REST, напротив, ограничен четырьмя возможными операциями, соответствующими четырем методам HTTP.
\end{itemize}

Как видно, в архитектура REST очень проста в плане использования. По виду пришедшего запроса сразу можно определить, что он делает, не разбираясь в форматах (в отличие от SOAP). Данные передаются без применения дополнительных слоев, поэтому REST считается менее ресурсоемким, поскольку не надо парсить запрос чтоб понять что он должен сделать и не надо переводить данные из одного формата в другой.

Архитектура REST позволяет серьезно упростить эту задачу. Конечно в реальности, того что описано не достаточно, ведь нельзя кому угодно давать возможность изменять информацию, то есть нужна еще авторизация и аутентификация. Но это достаточно просто разрешается при помощи различного типа сессий или просто HTTP Authentication.

Таким образом, в силу простоты использования, в качестве подхода взаимодействия в сети был выбран REST.