\subsection{Сравнение нативных приложений и ботов} 
\label{sec:analysis:botsvsnative}

Нельзя отрицать: чат-боты во многом привлекательны потому, что в них видят возможность сделать работу с компьютерами более естественной — вместо работы с графическим интерфейсом можно будет просто сказать, что вам нужно. До такого мы пока не дошли и дойдём лишь через некоторое время, но сегодняшние чат-боты показывают нам слабые проблески того будущего, в котором можно будет что-то сказать или написать, компьютер поймёт это и сделает, что требуется.

Способности чат-ботов через несколько лет, если сравнить их с нынешними, могут разочаровать — это подтвердит любой, кто имел дело с Alexa и Siri. Одно из безусловных препятствий к всеобщему проникновению ботов — это синтаксис, который нужно помнить для взаимодействия с ними. В конце концов развитие технологий обработки языка и искусственного интеллекта позволит решить эту проблему — компьютеру можно будет сказать почти что угодно, и он поймёт, о чём речь.

До этого пока далеко, но и нынешние боты могут быть полезными, хотя никакого искусственного интеллекта у них нет. Мы уже видим, что целый ряд повседневных задач можно обрабатывать путём переписки с автоматизированным агентом. Даже если этот бот не очень «умный» и требует знания некоторого синтаксиса, возможность делать дела посредством короткого обмена репликами с чат-ботом видится по-своему волшебной.

Простота разговорных интерфейсов может быть главным драйвером их распространения, но дискуссия о противостоянии ботов и приложений упускает кое-что из виду. Это контекст использования, и он, возможно, важнее чем простота. Большая доля привлекательности чат-ботов объясняется тем, что разговорные интерфейсы обладают уникальной возможностью интеграции в контекст, в котором мы уже находимся. Вот почему чат-боты особенно востребованы, когда цена смены контекста слишком высока.

Возможность дать команду в чате — это не только проще, чем сделать это через приложение. Она обеспечивает взаимодействие с минимумом отвлекающих факторов. По этой причине чат-ботам не обязательно нужно быть проще, чем аналогичные им приложения. Они должны соответствовать контексту, в котором вы что-то делаете.

Объясню, что имею в виду. У меня на кухне стоит Amazon Echo. Наверное, в первую очередь каждым утром я спрашиваю, какая погода. Могу я достать телефон и посмотреть прогноз на нём? Конечно, но пока я готовлю завтрак, гораздо проще сказать: «Алекса, какая сегодня погода?» Я не должен прерываться и «переключать режим».

Схожим образом боты в Slack позволяют делать всё что угодно в одной коллективной среде, что проще, нежели выполнение задач в разных интерфейсах и последующий сбор результатов в одно место. Интеграция вашего сервиса со Slack в виде бота снижает барьеры к принятию, потому что пользоваться сервисом можно будет в том же разговорном потоке, в котором уже ведутся дела.

Ставка на чат-ботов предполагает, что мы будем проводить всё больше времени в мессенджерах, и что пользоваться нужными сервисами при помощи ботов будет проще, чем открывать специальное приложение или сайт. В пользу такого развития событий уже говорит опыт Китая, в котором 650 млн человек проводят кучу времени в мессенджере WeChat: они не просто болтают с друзьями, но и потребляют широкий ассортимент услуг.

Это причина, по которой Facebook возлагает большие надежды на бот-платформу для Messenger. Продукт, который они скоро представят, скорее всего, позволит почти всем желающим поставлять контент или услуги через мессенджер. Компания идёт на этот шаг, рассчитывая на то, что в конечном итоге пользователи станут «жить» в Messenger — как в случае WeChat. Это такой манёвр, чтобы обойти iOS и Android как платформы для приложений, и сделать все сервисы доступными в Messenger.

Сначала дело может пойти медленно — пользователям потребуется время, чтобы привыкнуть к взаимодействию с ботами, но не трудно представить, что пользовательское поведение будет меняться по мере того, как элементов искусственного интеллекта в чат-платформах будет становиться больше, а грань между ботами и приложениями размоется.

Если боты победят, это произойдёт не по причине простоты использования, а из-за очевидной пользы от возможности встраивания в контекст, в котором мы уже работаем или общаемся. Мы привыкнем разговаривать с ботами в силу удобства.