\sectioncentered*{Определения и сокращения}
\label{sec:definitions}

В настоящей пояснительной записке применяются следующие определения и сокращения.
\\

\emph{Спецификация} -- документ, который желательно полно, точно и верифицируемо определяет требования, дизайн, поведение или другие характеристики компонента или системы, и, часто, инструкции для контроля выполнения этих требований \cite{istqb_specification}.

\emph{Веб-приложение} -- клиент-серверное приложение, в котором клиентом выступает браузер, а сервером -- веб-сервер.

\emph{Кроссплатформенность} -- способность программного обеспечения работать более чем на одной аппаратной платформе и (или) операционной системе.

\emph{Нативное программное средство} -- программное средство, специфичное для какой-либо платформы \cite{habr_crossplatform}.

\emph{Проприетарное программное обеспечение} -- программное обеспечение, являющееся частной собственностью авторов или правообладателей и не удовлетворяющее критериям свободного ПО: свобода запуска программы в любых целях, свобода адаптации программы для любых нужд, свобода распространения, свобода улучшений исходных кодов и публикации улучшений~\cite{free_software}.
\\

ВУЗ -- высшее учебное заведение.

ПС -- программное средство.

ПО -- программное обеспечение.

БД -- база данных.

СУБД -- система управления базами данных.

ЯП -- язык программирования.

API -- application programming interface (сетевой программный интерфейс).

UI -- user interface (пользовательский интерфейс).

ТЭО -- технико-экономическое обоснование.


\sectioncentered*{Введение}
\addcontentsline{toc}{section}{Введение}
\label{sec:introduction}


Мессенджер Telegram, придуманный Павлом Дуровым, становится все популярнее. Помимо того, что это удобный, надежно защищенный кросплатформенный мессенджер, технология API позволяет создавать ботов для приложения. Боты – это аккаунты, управляемые программами, которые откликаются на определенные команды.

Некоторые боты создаются для помощи пользователям, некоторые делаются лишь для развлечения. При этом постоянно увеличивается число ботов-помощников, которые могут быть использованы для решения тех или иных бизнес-задач.

Уже сегодня боты помогают пользователям решать совершенно разные задачи, будь то отслеживание погоды или конвертирование валют, прослушивание музыки или отслеживания посылок. На данном этапе боты развиваются в сторону нейронных сетей, чтобы обрабатывать больше данных и быть неотличимыми от людей. Сегодняшние приложения для слежения за финансами слишком перегружены разнообразными ненужными функциями, которые только отталкивают потенциального пользователя.

Целью настоящего дипломного проекта является разработка программного средства для слежения за финансами в виде бота для мессенджера Telegram. В основу ставится простота использования, понятность интерфейса и только полезные функции.

В пояснительной записке к дипломному проекту излагаются детали поэтапной разработки приложения для контроля за персональными расходами. В первом разделе приведены результаты анализа литературных источников по теме дипломного проекта, рассмотрены особенности существующих систем-аналогов, выдвинуты требования к проектируемому ПС. Во втором разделе приведено описание функциональности проектируемого ПС, представлена спецификация функциональных требований. В третьем разделе приведены детали проектирования и конструирования ПС. Результатом этапа конструирования является функционирующее программное средство. В четвертом разделе представлены доказательства того, что спроектированное ПС работает в соответствии с выдвинутыми требованиями спецификации. В пятом разделе приведены сведения по развертыванию и запуску ПС, указаны требуемые аппаратные и программные средства. Обоснование целесообразность создания программного средства с технико-экономической точки зрения приведено в шестом разделе. Итоги проектирования, конструирования программного средства, а также соответствующие выводы приведены в заключении.