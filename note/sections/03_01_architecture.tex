\subsection{Разработка архитектуры программного средства}
\label{sec:design:architecture}

Программный продукт представляет собой API-сервер, обрабатывающий
данные пользователей. Клиентским приложением является Telegram.

Принцип взаимодействия между сервером и Telegram основывается на концепции webhook.

Концепция WebHook проста. При старте приложение отправляет специальный запрос на сервис, который, при наступлении определенного события (в данном случае это сообщение от пользователя), отправляет POST запрос к нашему приложению, в теле которого содержатся все данные, необходимые для успешной обработки сообщения. Для пользователя WebHooks является способом получать ценную информацию в момент, когда произошло определенное событие, не опрашивая сервер постоянно, не получая взамен ничего ценного большую часть времени. 

Абсолютное большинство ресурсов сети Интернет реализованы с высокой степенью интерактивности, то есть предоставляют некоторые элементы, с которыми пользователь может взаимодействовать. Таким образом, клиентская часть приложения, помимо простого отображения информации, должна быть управляемой с помощью пользовательских воздействий. Выбранная в пункте \ref{sec:analysis:specification:language} программная платформа позволяет нам это реализовать.

Вследствие интерактивности появляется необходимость в хранилище\linebreakданных, которая обычно решается с помощью реляционных баз данных. В пункте \ref{sec:analysis:specification:language} указано, что в приложении, описываемом в данном дипломном проекте будет использоваться СУБД MongoDB, которая может запускаться на физически отдельной ЭВМ.

Таким образом, в конечном итоге, мы должны получить программный продукт, представляющий из себя API-сервер, который взаимодействует с клиентской частью с помощью протокола WebHook, принимающий сообщения от пользователя, производящий обработку, взаимодействие с базой данных, и возвращающий результат в виде ответного сообщения пользователю.
