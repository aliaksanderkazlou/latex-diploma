\sectioncentered*{Реферат}
\thispagestyle{empty}

% Зачем: чтобы можно было вывести общее число страниц.
% Добавляется единица, поскольку последняя страница -- ведомость.
\FPeval{\totalpages}{round(\getpagerefnumber{LastPage} + 1, 0)}

\noindent
\MakeUppercase{Программное средство «Чат-бот для контроля за \linebreak персональными расходами»}: дипломный проект / А. С. Козлов. -- Минск : БГУИР, 2018, -- п.з. -- \totalpages~с., чертежей (плакатов) -- 6 л. формата А1.
\vspace{\baselineskip}

Цель настоящего дипломного проекта состоит в разработке программной системы, предназначенной для упрощения процесса слежения за персональными расходами.

В процессе анализа предметной области были выделены основные аспекты процесса слежения за финансами. Был проведен сравнительный анализ чат-ботов и нативных приложений, отмечены слабые и сильные стороны каждого из подходов. Кроме того, рассмотрены существующие аналоги, применяемые для облегчения контроля за финансами, выделены общие недостатки. Выработаны функциональные и нефункциональные требования.

Была разработана архитектура программной системы, для каждой ее составной части было проведено разграничение реализуемых задач, проектирование, уточнение используемых технологий и собственно разработка. Были выбраны наиболее современные средства разработки, широко применяемые в индустрии. 

Полученные в ходе технико-экономического обоснования результаты о прибыли для разработчика, пользователя, уровень рентабельности, а также экономический эффект доказывают целесообразность разработки про\-екта.